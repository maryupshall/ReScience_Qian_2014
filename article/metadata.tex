% DO NOT EDIT - automatically generated from metadata.yaml

\def \codeURL{https://github.com/mupsh/ReScience_Levitan_2014}
\def \codeDOI{}
\def \dataURL{}
\def \dataDOI{}
\def \editorNAME{}
\def \editorORCID{}
\def \reviewerINAME{}
\def \reviewerIORCID{}
\def \reviewerIINAME{}
\def \reviewerIIORCID{}
\def \dateRECEIVED{}
\def \dateACCEPTED{}
\def \datePUBLISHED{}
\def \articleTITLE{ReScience (R)evolution}
\def \articleTYPE{Replication}
\def \articleDOMAIN{Computation Neuroscience}
\def \articleBIBLIOGRAPHY{bibliography.bib}
\def \articleYEAR{2020}
\def \reviewURL{}
\def \articleABSTRACT{Dopamine neurons in freely moving rats often fire behaviorally relevant high-frequency bursts, but depolarization block limits the maximum steady firing rate of dopamine neurons in vitro to ∼10 Hz. Using a reduced model that faithfully reproduces the sodium current measured in these neurons, we show that adding an additional slow component of sodium channel inactivation, recently observed in these neurons, qualitatively changes in two different ways how the model enters into depolarization block. First, the slow time course of inactivation allows multiple spikes to be elicited during a strong depolarization prior to entry into depolarization block. Second, depolarization block occurs near or below the spike threshold, which ranges from -45 to -30 mV in vitro, because the additional slow component of inactivation negates the sodium window current. In the absence of the additional slow component of inactivation, this window current produces an N-shaped steady-state current-voltage (I-V) curve that prevents depolarization block in the experimentally observed voltage range near -40 mV. The time constant of recovery from slow inactivation during the interspike interval limits the maximum steady firing rate observed prior to entry into depolarization block. These qualitative features of the entry into depolarization block can be reversed experimentally by replacing the native sodium conductance with a virtual conductance lacking the slow component of inactivation. We show that the activation of NMDA and AMPA receptors can affect bursting and depolarization block in different ways, depending upon their relative contributions to depolarization versus to the total linear/nonlinear conductance.}
\def \replicationCITE{Original article (Mathematical analysis of depolarization block mediated by slow inactivation of fast sodium channels in midbrain dopamine neurons., Levitan, E. S., Canavier, C. C., Tucker, K. R., Qian, K., \& Yu, N., Journal of Neurophysiology}
\def \replicationBIB{Levitan2014}
\def \replicationURL{https://www.physiology.org/doi/full/10.1152/jn.00578.2014}
\def \replicationDOI{10.1152/jn.00578.2014}
\def \contactNAME{Mary Upshall}
\def \contactEMAIL{mupsh059@uottawa.ca}
\def \articleKEYWORDS{Python, depolarization block, dopaminergic neurons}
\def \journalNAME{ReScience C}
\def \journalVOLUME{4}
\def \journalISSUE{1}
\def \articleNUMBER{}
\def \articleDOI{}
\def \authorsFULL{Mary Upshall and Aaron R. Shifman}
\def \authorsABBRV{M. Upshall and A.R. Shifman}
\def \authorsSHORT{Upshall and Shifman}
\title{\articleTITLE}
\date{}
\author[1,2,3,\orcid{0000-0003-0376-843X}]{Mary Upshall}
\author[1,2,3,\orcid{0000-0003-2140-7590}]{Aaron R. Shifman}
\affil[1]{Department of Biology, University of Ottawa, Ottawa, Ontario, Canada}
\affil[2]{uOttawa Brain and Mind Research Institute, Ottawa, Ontario, Canada}
\affil[3]{Center for Neural Dynamics, University of Ottawa, Ottawa, Ontario, Canada}
